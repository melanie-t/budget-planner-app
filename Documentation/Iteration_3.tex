\documentclass[12pt]{article}

\pagestyle{empty}
\setcounter{secnumdepth}{2}

\topmargin=0cm
\oddsidemargin=0cm
\textheight=22.0cm
\textwidth=16cm
\parindent=0cm
\parskip=0.15cm
\topskip=0truecm
\raggedbottom
\abovedisplayskip=3mm
\belowdisplayskip=3mm
\abovedisplayshortskip=0mm
\belowdisplayshortskip=2mm
\normalbaselineskip=12pt
\normalbaselines

\begin{document}

\vspace*{0.5in}
\centerline{\bf\Large Iteration 3 - Test Document}

\vspace*{0.5in}
\centerline{\bf\Large Team PA-PI-a}

\vspace*{0.5in}
\centerline{\bf\Large 8 April 2018}

\vspace*{1.5in}
\begin{table}[htbp]
\caption{Team}
\begin{center}
\begin{tabular}{|r | c|}
\hline
Name & ID Number \\
\hline\hline
Melanie Taing & 40009850 \\
Laurie Gagnon & 22943433 \\
Wayne Yiel Leung & 26586988 \\
Jordan Rutty & 27300107 \\
Michael Foo & 40000225 \\
Pierre-Andre Leger & 40004010 \\
Colin Greczkowski & 40001600 \\
\hline
\end{tabular}
\end{center}
\end{table}

\clearpage

\tableofcontents
\clearpage

\section{Introduction}

{\it
The introduction of the document provides an overview of the entire document,
briefly introducing what are its goals, and what information is to be found in it.
}

\section{Test Plan}

{\it
Describe what forms of testing you plan to do (unit, subsystem, integration),
describe briefly the schedule and resources for testing,
and
how you designed your test cases.

Indicate which qualities (from requirements) were tested and which qualities were not tested.
}

\subsection{System Level Test Cases}

{\it
All test cases for testing at the system level.
}

\subsubsection{Test Case 1} \label{tc:1}

\noindent
{\bf Purpose}\\
State the purpose of the test.
Indicate which requirement and which aspect of that requirement is being tested.

\noindent
{\bf Input Specification}\\
State the context for the test in terms of system state.
State the input test data. You may need to mention operations invoked as well as data for the operation.
You can cross-reference to actual file data specified in an appendix.

\noindent
{\bf Expected Output}\\
State the expected system response and output.
You can cross-reference to actual file data specified in an appendix.

\noindent
{\bf Traces to Use Cases}\\
State which requirements (at the level of use case and scenario) are tested by this test case.

\subsection{Subsystem Level Test Cases}

{\it
All test cases for testing at the subsystem level.
}

{\it
One subsection per subsystem
}

\subsubsection{Subsystem X}

\subsection{Unit Test cases}

{\it
All test cases for testing at the unit level.
}

{\it
One subsection per unit
}

\subsubsection{Unit X}

\section{Test Results}

{\it
List the tests, indicating which passed and which did not pass.
List requirements indicating the percentage of tests that passed for that requirement.
}

\section{References}

\appendix

\section{Description of Input Files}

Describe/include test data from input files.

\section{Description of Output Files}

Describe/include test expected output that are output files.

\end{document}
