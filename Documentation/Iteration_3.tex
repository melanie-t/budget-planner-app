\documentclass[12pt]{article}
\usepackage{float}
\usepackage{graphicx}
\usepackage{tabularx}
\usepackage[font=small,labelfont=bf]{caption}
\usepackage{xcolor}
\usepackage{hyperref}

\pagestyle{empty}
\setcounter{tocdepth}{4}
\setcounter{secnumdepth}{4}

\topmargin=0cm
\oddsidemargin=0cm
\textheight=22.0cm
\textwidth=16cm
\parindent=0cm
\parskip=0.15cm
\topskip=0truecm
\raggedbottom
\abovedisplayskip=3mm
\belowdisplayskip=3mm
\abovedisplayshortskip=0mm
\belowdisplayshortskip=2mm
\normalbaselineskip=12pt
\normalbaselines

\begin{document}

\vspace*{0.5in}
\centerline{\bf\Large Iteration 3 - Test Document}

\vspace*{0.5in}
\centerline{\bf\Large Team PA-PI-a}

\vspace*{0.5in}
\centerline{\bf\Large 8 April 2018}

\vspace*{1.5in}
\begin{table}[htbp]
\caption{Team}
\begin{center}
\begin{tabular}{|r | c|}
\hline
Name & ID Number \\
\hline\hline
Melanie Taing & 40009850 \\
Laurie Gagnon & 22943433 \\
Wayne Yiel Leung & 26586988 \\
Jordan Rutty & 27300107 \\
Michael Foo & 40000225 \\
Pierre-Andre Leger & 40004010 \\
Colin Greczkowski & 40001600 \\
\hline
\end{tabular}
\end{center}
\end{table}

\clearpage

\tableofcontents
\clearpage

\section{Introduction}
The purpose of this document is to gather all information necessary for testing of the MyMoney application. This document describes the testing approach and overall framework that will be used to test the MyMoney application.\\

The following pages will identify the requirements that will be tested, the testing strategy used, the test cases and their results, and the description of input files.

\section{Test Plan}

{\it
Describe what forms of testing you plan to do (unit, subsystem, integration),
describe briefly the schedule and resources for testing,
and
how you designed your test cases.

Indicate which qualities (from requirements) were tested and which qualities were not tested.
}

\subsection{System Level Test Cases}
{\it
All test cases for testing at the system level.}

\noindent
{\bf Purpose}\\
State the purpose of the test.
Indicate which requirement and which aspect of that requirement is being tested.

\noindent
{\bf Input Specification}\\
State the context for the test in terms of system state.
State the input test data. You may need to mention operations invoked as well as data for the operation.
You can cross-reference to actual file data specified in an appendix.

\noindent
{\bf Expected Output}\\
State the expected system response and output.
You can cross-reference to actual file data specified in an appendix.

\noindent
{\bf Traces to Use Cases}\\
State which requirements (at the level of use case and scenario) are tested by this test case.
\clearpage

\def\arraystretch{1.5}%
\begin{table}[htbp]
\centering
\caption{Template Test Case}
\label{UT-1}
\begin{tabularx}{\textwidth}{ | l | X |}
\hline
\textbf{Test Case Number}      &  UT-1                         \\ \hline
\textbf{Test Case Description}    &  This test case is used to ensure the generated puzzle board has the same dimensions as the input width and height                \\ \hline
\textbf{Input}         & 	\begin{enumerate}
	\item None - Default 8-8 board size
	\item  Width/height from "input.txt" file. 
\end{enumerate} \\ \hline

\textbf{Expected Output}     & \begin{enumerate}
	\item "OK" - Test executed successfully.
\end{enumerate} \\ \hline
\textbf{Expected Post-Conditions}           &  The system responds to the presence or absence of the input vector and outputs a success message upon test successful completion in "output.txt", along with a time-stamp containing the test's execution time and date.                    \\ \hline
\textbf{Execution History}   &  \begin{enumerate}
	\item 05/04/2018 | Tester's name | Executed test successfully.
\end {enumerate} \\ \hline
\end{tabularx}
\end{table}
\clearpage


\subsection{Subsystem Level Test Cases}

{\it
All test cases for testing at the subsystem level.
}

{\it
One subsection per subsystem
}

\subsubsection{Subsystem X}

\subsection{Unit Test cases}

{\it
All test cases for testing at the unit level.
}

{\it
One subsection per unit
}


\subsubsection{Unit Test Case 1} \label{tc:1}
\def\arraystretch{1.5}%
\begin{table}[htbp]
\centering
\caption {UT-1}
\label{UT-1}
\begin{tabularx}{\textwidth}{ | l | X |}
\hline
\textbf{Test Case Number}      &  UT-1                         \\ \hline
\textbf{Test Case Description}    &  This test case is used to ensure that transactions are properly saved or updated to their repository                \\ \hline
\textbf{Input}         & 	\begin{enumerate}
	\item A Transaction object populated with generic data
          \item A second Transaction object with the ID of the first one.
	\item A test transaction database.
\end{enumerate} \\ \hline

\textbf{Expected Output}     & \begin{enumerate}
	\item Transaction details are printed to console.
\end{enumerate} \\ \hline
\textbf{Expected Post-Conditions}           &  A transaction database is created and a transaction is inserted. The balance of this transaction is then updated to a new value.                   \\ \hline
\textbf{Execution History}   &  \begin{enumerate}
	\item 04/03/2018 | Colin Greczkowski | Executed test successfully.
\end {enumerate} \\ \hline
\end{tabularx}
\end{table}
\clearpage

\subsubsection{Unit Test Case 2} \label{tc:2}
\def\arraystretch{1.5}%
\begin{table}[htbp]
\centering
\caption {UT-2}
\label{UT-2}
\begin{tabularx}{\textwidth}{ | l | X |}
\hline
\textbf{Test Case Number}      &  UT-2                         \\ \hline
\textbf{Test Case Description}    &  This test verifies that the deleteItem method works as intended, and deletes a Transaction record for a given ID                \\ \hline
\textbf{Input}         & 	\begin{enumerate}
          \item A generic account ID
	\item A Transaction object populated with generic data, associated to the generic account.
	\item A test transaction database.
\end{enumerate} \\ \hline

\textbf{Expected Output}     & \begin{enumerate}
	\item "Delete Transaction 1"
\end{enumerate} \\ \hline
\textbf{Expected Post-Conditions}           & The test transaction database should be empty.                \\ \hline
\textbf{Execution History}   &  \begin{enumerate}
	\item 04/07/2018 | Colin Greczkowski | Executed test successfully.
\end {enumerate} \\ \hline
\end{tabularx}
\end{table}
\clearpage

\subsubsection{Unit Test Case 3} \label{tc:3}
\def\arraystretch{1.5}%
\begin{table}[htbp]
\centering
\caption {UT-3}
\label{UT-3}
\begin{tabularx}{\textwidth}{ | l | X |}
\hline
\textbf{Test Case Number}      &  UT-3                         \\ \hline
\textbf{Test Case Description}    &  This test case is used to make sure all Transactions associated to an account are properly purged from the repository.                \\ \hline
\textbf{Input}         & 	\begin{enumerate}
          \item A generic account ID
	\item Two Transaction objects populated with generic data, associated to the generic account.
	\item A test transaction database.
\end{enumerate} \\ \hline

\textbf{Expected Output}     & \begin{enumerate}
	\item "Delete Transaction 1"
           \item "Delete Transaction 2"
\end{enumerate} \\ \hline
\textbf{Expected Post-Conditions}           & The test transaction database does not contain the two transactions that had the generic account ID.                \\ \hline
\textbf{Execution History}   &  \begin{enumerate}
	\item 04/03/2018 | Colin Greczkowski | Executed test successfully.
\end {enumerate} \\ \hline
\end{tabularx}
\end{table}
\clearpage

\subsubsection{Unit Test Case 4} \label{tc:1}
\def\arraystretch{1.5}%
\begin{table}[htbp]
\centering
\caption {UT-4}
\label{UT-4}
\begin{tabularx}{\textwidth}{ | l | X |}
\hline
\textbf{Test Case Number}      &  UT-4                         \\ \hline
\textbf{Test Case Description}    &  This tests the RepositoryContainer's ability to save a variety of types of objects (Transactions, Accounts, Budgets).                \\ \hline
\textbf{Input}         & 	\begin{enumerate}
          \item Test Transaction, Budget and Account Databases
	\item A test Transaction
	\item A test Account
	\item A test Budget
\end{enumerate} \\ \hline

\textbf{Expected Output}     & \begin{enumerate}
	\item The test transaction's details are printed to console.
\end{enumerate} \\ \hline
\textbf{Expected Post-Conditions}           & The account,transaction and budget items are saved to their respective test databases. Balances are updated correctly.                \\ \hline
\textbf{Execution History}   &  \begin{enumerate}
	\item 04/07/2018 | Colin Greczkowski | Executed test successfully.
\end {enumerate} \\ \hline
\end{tabularx}
\end{table}
\clearpage


\subsubsection{Unit Test Case 5} \label{tc:1}
\def\arraystretch{1.5}%
\begin{table}[htbp]
\centering
\caption {UT-5}
\label{UT-5}
\begin{tabularx}{\textwidth}{ | l | X |}
\hline
\textbf{Test Case Number}      &  UT-5                         \\ \hline
\textbf{Test Case Description}    &  TO BE COMPLETED: BudgetContainer test                \\ \hline
\textbf{Input}         & 	\begin{enumerate}
	\item A Budget object populated with generic data
	\item A test Account database with transactions
	\item A test Budget database
\end{enumerate} \\ \hline

\textbf{Expected Output}     & \begin{enumerate}
	\item The test budget's details are printed to console.
\end{enumerate} \\ \hline
\textbf{Expected Post-Conditions}           & A budget database is created and a budget is inserted. The recorded budget amount is updated according to transactions made across all accounts for that budget.            \\ \hline
\textbf{Execution History}   &  \begin{enumerate}
	\item 04/08/2018 | Melanie Taing | Not executed
\end {enumerate} \\ \hline
\end{tabularx}
\end{table}
\clearpage



%Account repository test case
\begin{table}[htbp]
\centering
\label{UT-6}
\begin{tabularx}{\textwidth}{ | l | X |}
\hline
\textbf{Test Case Number}      &  UT-6                      \\ \hline
\textbf{Test Case Description}    &  This test case is used to ensure created accounts are saved in the database. Also it verifies that accounts can be deleted from it.                 \\ \hline
\textbf{Input}         & 	\begin{enumerate}
  
\item An account database
\item An account repository
\item An account with non-null values for \textit{nickname}, \textit{bankName} and a non-negative value for \textit{balance}
  
\end{enumerate} \\ \hline

\textbf{Expected Output}     & \begin{enumerate}
\item Tuples in Account repository test before delete: 2 
\item Tuples in Account repository test after delete: 1 
\item Current items loaded in repo:1 
\item 1

\end{enumerate} \\ \hline
\textbf{Expected Post-Conditions} & The system has a single account in the account database. \\
\hline
\textbf{Execution History}   &  \begin{enumerate}
	\item 04/07/2018 | Wayne Yiel Leung | Executed test failed.
\end {enumerate} \\ \hline
\end{tabularx}
\end{table}




\section{Test Results}

{\it
List the tests, indicating which passed and which did not pass.
List requirements indicating the percentage of tests that passed for that requirement.
}

\section{References}

\appendix

\section{Description of Input Files}

Describe/include test data from input files.

\end{document}
